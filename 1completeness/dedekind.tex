\graphicspath{{notes/1completeness/}}

\title{Dedekind Cuts}
\maketitle

\thispagestyle{empty}

\subsection{The Set $\R$ of Real Numbers}

\begin{axioms}[Field]\label{axiom:field}
Consider a set $\F$ with two binary operations $+$ and $\cdot$. We say that $\F$ is an \emph{ordered field} if, for all $a,b,c\in\F$,
\begin{itemize}
\item[A0] $a+b\in\F$ \ (closure)
\item[A1] $a+(b+c)=(a+b)+c$ \ (associativity)
\item[A2] $a+b=b+a$ \ (commutativity)
\item[A3] $\exists 0\in\F$ such that $a+0=a$ \ (identity)
\item[A4] Given $a\in\F$, $\exists -a\in\F$ such that $a+(-a)=0$ \ (inverse)
\item[M0] $ab\in\F$ \ (closure)
\item[M1] $a(bc)=(ab)c$ \ (associativity)
\item[M2] $ab=ba$ \ (commutativity)
\item[M3] $\exists 1\in\F$ such that $a\cdot 1=a$ \ (identity)
\item[M4] Given $a\in\F\setminus\{0\}$, $\exists a^{-1}\in\F$ such that $aa^{-1}=1$ \ (inverse)
\item[D] $a(b+c)=ab+ac$ \ (distributivity)
\end{itemize}
\end{axioms}

\begin{axioms}[Ordered Field]\label{axiom:ord}
An \emph{ordered field} is a field $\F$ together with a binary relation $\le$ such that, for all $a,b,c\in\F$,
\begin{itemize}
	\item[O1] $a\le b$ or $b\le a$
	\item[O2] $a\le b$ and $b\le a\implies a=b$
	\item[O3] $a\le b$ and $b\le c\implies a\le c$
	\item[O4] $a\le b\implies a+c\le b+c$
	\item[O5] $a\le b$ and $0\le c\implies ac\le bc$
\end{itemize}
\end{axioms}

\subsection{A Development of $\R$}

\begin{enumerate}
	\item (Define $\N\cup\{0\}$ and $\Z$) \ Either use Peano's Axioms, or, if you want an explicit set-theoretic construction, define:
	\[0:=\emptyset,\qquad n+1:=n\cup \{n\}\]
	E.g. $1:=\{0\}=\{\emptyset\},\qquad 2:=\{0,1\}=\{\emptyset,\{\emptyset\}\},\qquad 3=\{0,1,2\}$,\qquad etc.
	\item (Define $\Q$) \ Consider equivalence classes of pairs:
	\[\Q=\quotient{\Z\times\N}{\sim}\quad\text{where}\quad(p,q)\sim(r,s)\iff ps=qr\]
	This is usually written $\frac pq=\frac rs$. One now carefully defines $+,\cdot$ and $\le$ on these equivalence classes using such concepts for the integers.
	\item (Define $\R$) \ Let $\alpha$ be a non-empty proper subset of $\Q$ with the following properties:
	\begin{itemize}
  	\item If $r\in\alpha$ and $s\in\Q$ with $s<r$ then $s\in\alpha$.
  	\item $\alpha$ has no largest rational number ($\max\alpha$ makes no sense).
	\end{itemize}
	Such sets $\alpha$ are called \emph{Dedekind cuts}. Now define $\R$ to be the set of Dedekind cuts!
\end{enumerate}

This construction probably seems weird, though at least it is explicit. For example, the Dedekind cut
\[\alpha=\{x\in\Q:x<4\}\]
corresponds to the real number 4. The real number $\sqrt 2$ corresponds to the Dedekind cut
\[\beta=\{x\in\Q:x<0\text{ or }x^2<2\}\]
Colloquially, the real number $\alpha$ corresponds to the Dedekind cut `all rationals less than $\alpha$.' In particular, note that every Dedekind cut is bounded above.\\

\begin{defn}
If $\alpha,\beta\in\R$, define
\[\alpha+\beta:=\{a+b:a\in\alpha,b\in\beta\}\]
\end{defn}

\begin{description}
\begin{itemize}
	\item[A0] If $\alpha,\beta\in\R$. Certainly $\alpha+\beta$ is a non-empty, proper subset of $\Q$ (let $x\in\Q$ be an upper bound for both $\alpha,\beta$, then $2x$ is an upper bound for $\alpha+\beta$).\\
	Let $a+b\in\alpha+\beta$ where $a\in\alpha,b\in\beta$ and let $y\in\Q$ satisfy $y<a+b$. Then $y-a<b$ and so $y-a\in\beta$ (it's rational!). But then $y=a+(y-a)\in\alpha+\beta$ and we're done. 
	\item[A1] Associativity of addition is trivial: since addition in $\Q$ is associative,
	\[\alpha+\beta+\gamma=\{a+b+c:a\in\alpha,b\in\beta,c\in\gamma\}\]
	\item[A2] Commutativity of addition is immediate since it is also in $\Q$.
	\item[A3] Let $0=\{x\in\Q:x<0\}$ and let $\alpha\in\R$ be a Dedekind cut. Then
	\[\alpha+0=\{a+x\in\Q:a\in\alpha,x<0\}\]
	Clearly $\alpha+0\subseteq\alpha$. For the converse, observe that if $a\in\alpha$, then $a-1\in\alpha$ and so $a=(a-1)+1\in\alpha+0$. It follows that $\alpha+0=\alpha$. 
	\item[A4] Given $\alpha\in\R$, define\
	\[\alpha^\circ=\{x\in\Q:\forall a\in\alpha, x<-a\}\]
	and
	\[-\alpha=\begin{cases}
	\alpha^\circ\setminus\{\max\alpha^\circ\}&\text{ if the maximum exists}\\
	\alpha^\circ&\text{otherwise}
	\end{cases}\]
	This is clearly a set of rational numbers which is non-empty: suppose that $M_\alpha$ is an upper bound for $\alpha$, then
	\[\forall a\in \alpha, a<M+1\implies -(M+1)<-a\implies -(M+1)\in -\alpha\]
	(choosing $M+1$ guarantees that $-(M+1)\noteq\max\alpha^\circ$\ldots)\\
	It should also be clear that $-\alpha$ is closed downwards and is therefore a Dedekind cut.\\
	It should be completely clear that if $x\in-\alpha$ and $a\in\alpha$ that $a+x<0$, whence $\alpha+(-\alpha)\subseteq 0$. For the converse, suppose $q<0$ and let $a\in\alpha$ be such that $q-a\in\-\alpha$ (such $a$ exists by induction!). But then $q=a+(q-a)\in\alpha+(-\alpha)$.
\item[M0] $ab\in\F$ \ (closure)
\item[M1] $a(bc)=(ab)c$ \ (associativity)
\item[M2] $ab=ba$ \ (commutativity)
\item[M3] $\exists 1\in\F$ such that $a\cdot 1=a$ \ (identity)
\item[M4] Given $a\in\F\setminus\{0\}$, $\exists a^{-1}\in\F$ such that $aa^{-1}=1$ \ (inverse)
\item[D] $a(b+c)=ab+ac$ \ (distributivity)
\end{itemize}
\end{description}

It remains to \emph{prove} all the axioms of a complete ordered field. We omit the details, though here is a rough overview.
\begin{itemize}
  \item Define the ordering of Dedekind cuts by
	\[\alpha\le\beta\iff\alpha\subseteq\beta\]
	One can now prove axioms O1--O3 and that the ordering corresponds to that of $\Q$.
	\item Define addition by
	\[\alpha+\beta:=\{r_1+r_2:r_1\in\alpha,r_2\in\beta\}\]
	Now prove axioms A0--A4 and O4 (with careful definition of $-\alpha$: this is quite hard\ldots).
	\item Multiplication is horrible: if $\alpha,\beta\ge 0$ then
	\[\alpha\beta:=\{ab:a\ge 0,a\in \alpha,b\ge 0,b\in \beta\}\cup\{q\in\Q:q<0\}\]
	which is then extended to a definition when at least one of $\alpha,\beta<0$. Once can then prove axioms M0--M4, O5 and D.
	\item The completeness axiom must be proved, but it comes essentially for free! If $A\subseteq\R$ (so that $A$ is a set of Dedekind cuts!), then the supremum of $A$ is
	\[\sup A=\bigcup\limits_{\alpha\in A}\alpha\]
\end{itemize}
