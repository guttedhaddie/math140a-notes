\graphicspath{{notes/3series/asy/}}

\pagestyle{empty}

\boldsubsection{Decimal Expansions of Real Numbers}

We are typically introduced to decimals in elementary mathematics; for many in grade-school they become a working \emph{definition} of the real numbers. But what are they?

\begin{defn*}{}{}
	A \emph{non-negative decimal} $d_0.d_1d_2d_3\cdots$ is an infinite series of the form
	\[
		d_0+\sum_{n=1}^\infty d_n\,10^{-n}
		\ \text{ where }\ d_n\in\N_0 \ \text{ and }\ 
		\forall n\ge 1\implies d_n\le 9
	\]
	Let $x$ be a non-negative real number. Its \emph{decimal expansion} $D(x)$ is the decimal series arising from inductively defined sequences $(d_n)_{n=0}^\infty$ and $(R_n)_{n=0}^\infty$:
	\begin{gather*}
		\begin{cases}
			d_0=\lfloor x\rfloor,&d_{n+1}=\lfloor 10R_n\rfloor\\
			R_0=x-d_0,&R_{n+1}=10R_n-d_{n+1}
		\end{cases}
	\end{gather*}
	where we use the \emph{floor} function $\lfloor x\rfloor =\max\{n\in\Z:n\le x\}$.\smallbreak
	The decimal expansion of $x<0$ is negative that of $\nm x=-x$.
\end{defn*}

\begin{examples*}{}{}
	\exstart If $x=\frac{27}{20}$, then,
  \[
  	\def\arraystretch{1.4}
  	\begin{array}{l||c|c|c|c|c|c}
  		n & 0 & 1 & 2 & 3 & 4 & \cdots \\\hline
  		d_n & \left\lfloor\frac{27}{20}\right\rfloor=1 
  			& \left\lfloor\frac{70}{20}\right\rfloor=3 
  			& \left\lfloor\frac{10}{2}\right\rfloor=5 
  			& \left\lfloor 0\right\rfloor=0 
  			& 0 & \cdots \\\hline
 	 		R_n & \frac 7{20} & \frac{1}{2} & 0 & 0 & 0 & \cdots
  	\end{array}
  \]
  Both sequences continue with zeros and we obtain the terminating decimal $D(\frac{27}{20})=1.35$.
  
  \begin{enumerate}\setcounter{enumi}{1}
		\item If $x=\frac 13$, then,
		\[
			\def\arraystretch{1.4}
			\begin{array}{l||c|c|c|c|c}
	  		n & 0 & 1 & 2 & 3 & \cdots\\\hline
	  		d_n & \left\lfloor\frac{1}{3}\right\rfloor=0 
	  			& \left\lfloor\frac{10}3\right\rfloor=3 
	  			& \left\lfloor\frac{10}3\right\rfloor=3 
	  			& \left\lfloor\frac{10}3\right\rfloor=3 
	  			& \cdots \\\hline
	  		R_n & \frac 13 & \frac 13 & \frac 13 & \frac 13 & \cdots
	  	\end{array}
	  \]
	  By induction, all $R_n=\frac 13$ and we recover the periodic decimal $D(\frac 13)=0.33333\cdots$.
	  
	  \item If $x=\frac 17$, then,
	  \[
	  	\def\arraystretch{1.4}
	  	\begin{array}{l||c|c|c|c|c|c|c|c|c}
	  		n & 0 & 1 & 2 & 3 & 4 & 5 & 6 & 7 & \cdots \\\hline
	  		d_n & 0 & \left\lfloor\frac{10}7\right\rfloor=1 
	  			& \left\lfloor\frac{30}7\right\rfloor=4 
	  			& \left\lfloor\frac{20}7\right\rfloor=2 
	  			& \left\lfloor\frac{60}7\right\rfloor=8 
	  			& \left\lfloor\frac{40}7\right\rfloor=5 
	  			& \left\lfloor\frac{50}7\right\rfloor=7 
	  			& 1 & \cdots \\\hline
	  		R_n & \frac 17 & \frac 37 & \frac 27 & \frac 67 & \frac 47 
	  			& \frac 57 & \frac 17 & \frac 37 & \cdots
	   \end{array}
	  \]
	  Since $R_6=R_0$, both sequences will repeat: $R_{n+6}=R_n$ and $d_{n+6}=d_n$. We recover the \emph{period-six} decimal $D(\frac 17)=0.142857142857\cdots$.
	\end{enumerate}
\end{examples*}


\goodbreak


In the main result, we check that the decimal expansion is well-defined and that it behaves as expected. We also give two well-known properties of decimal representations.

\begin{thm*}{}{}
	Let $x\in\R^+_0$ have decimal expansion $D(x)=\sum\limits_{n=0}^\infty d_n\,10^{-n}$. Then:
	\begin{enumeratea}
	  \item $D(x)$ is a decimal: each $d_n\in\{0,1,2,\ldots,9\}$ whenever $n\ge 1$.
	  \item $D(x)$ converges to $x$.
	  \item The sequence $(d_n)$ is \emph{eventually periodic} if and only if $x\in\Q^+_0$.
	  \item $x$ equals a unique decimal series, except when $D(x)=d_0.d_1\cdots d_m$ terminates ($d_m\neq 0$). In such a case there is a second decimal representation:
	  \[
	  	x=D(x) =d_0.d_1\cdots d_m =d_0.d_1\cdots d_{m-1}\hat{d_m}99999\cdots %=\sum_{n=0}^md_n\,10^{-n}=\sum_{n=0}^{m-1}d_n\,10^{-n}+(d_m-1)10^{-m}+9\sum_{n=m+1}^\infty 10^{-n}
	  \]
	  where $\hat{d_m}=d_m-1$. Otherwise said, we subtract 1 from the final non-zero term and insert an infinite string of 9's.
	\end{enumeratea}
\end{thm*}

\begin{examples*}{}{}
	\exstart Part (c) explains why so many people enjoy the challenge of memorizing the digits of $\pi$: since $\pi$ is irrational, the pattern never repeats. 
	\begin{enumerate}\setcounter{enumi}{1}
	  \item Also referencing part (c), we explicitly evaluate a \emph{period-three} decimal using geometric series:
	  \begin{align*}
	  	3.1279279279279\cdots&=\frac{31}{10}+\frac{279}{10000}\sum_{n=0}^\infty 1000^{-n}
	  		=\frac{31}{10}+\frac{279}{10000}\cdot\frac 1{1-\frac 1{1000}}\\
	  	&=\frac{31}{10}+\frac{279}{9990} =\frac{1736}{555}
	  \end{align*}
	  \item Here are two examples of part (d): 
	  \[
	  	1=0.99999\cdots\qquad 27.164=27.1639999\cdots
	  \]
	\end{enumerate}
\end{examples*}


\begin{exercises*}{}
	\exstart Compute the decimal expansion of $\frac{32}{13}$.
	
	\begin{enumerate}\setcounter{enumi}{1}
  	\item Prove all parts of the Theorem. Here are some hints:
		\begin{itemize}
	  	%\item Use a series test\ldots
	  	%\item Prove by induction: it should be obvious that $0\le R_0<1$, etc.
	  	\item[(a)] Let $E_n=x-\sum\limits_{k=0}^nd_k\,10^{-k}$. Prove by induction that $R_n=10^nE_n$ and conclude $\lim E_n=0$.
	  	\item[(c)] A decimal is eventually periodic with period $r$ if
	  	\[
	  		d_0.d_1\cdots d_md_{m+1}\cdots d_{m+r}d_{m+1}\cdots d_{m+r}\cdots 
	  		=\sum_{k=0}^md_k\, 10^{-k}
	  		+\left(\sum_{j=1}^rd_{m+j}\,10^{-m-j}\right)
	  		\sum_{l=0}^\infty 10^{-rl}
	  	\]
	  	Convince yourself that this is rational. For the converse, is $x=\frac pq$ is rational number observe that there are only \emph{finitely many} possible values for the remainders $R_n=\frac aq$.
	  	\item[(d)] If $d_0.d_1d_2\cdots=c_0.c_1c_2\cdots$, let $m$ be minimal such that $c_m<d_m$\ldots
		\end{itemize}
		
	\item\begin{enumerate}
	  \item Can you find a simple way to describe all the real numbers $x$ for which $D(x)$ is terminating? Prove your assertion.\par
	  (\emph{Hint: What form can the denominator of $R_n$ take if $x=\frac pq$?})
	  
	  \item Given a rational number $x=\frac pq$ in lowest terms with $q\in\N$, what is the \emph{largest} possible eventual period of $D(x)$? Explain.
	\end{enumerate}
	
		\item Similar analyses can be done for other representations of real numbers. For instance, by replacing 10 with 3 in the definition, we obtain the \emph{ternary} (base-3) expansion of a real number
		\[
			T(x) =[t_0.t_1t_2\cdots]_3
			=\sum_{n=0}^\infty t_n\,3^{-n}
			\ \text{ where }\ 
			t_n\in\{0,1,2\}
			\ \text{ whenever }\
			n\ge 1
		\]
		For example, $[0.1]_3=\frac 13$ and $[0.12]_3=\frac 13+\frac 2{3^2}=\frac 59$.
		\begin{enumerate}
	  	\item Compute $[0.02020202\cdots]_3$ and find the ternary representations of $\frac 12$ and $\frac 15$.
	  	\item Read over the Theorem. How can we modify its claims for ternary representations?
	  	\item Describe all real numbers $x$ whose ternary representation terminates. More generally, describe all real numbers $x$ whose $n$-ary (base-$n$) representation terminates.\par
	  	(\emph{The major take-away is that there is nothing special about base-10. Computers typically use base-2, 8 or 16; the ancient Babylonians used base-60. Modern humans likely settled on decimals because because we're blessed with 10 fingers\ldots})
		\end{enumerate}
	\end{enumerate}
\end{exercises*}

