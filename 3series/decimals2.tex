\graphicspath{{notes/3series/asy/}}

\pagestyle{empty}

\boldsubsection{Solutions to Decimals Exercises}

\begin{enumerate}
  \item If $x=\frac{32}{13}$, we obtain
  \[
  	\def\arraystretch{1.4}
  	\begin{array}{l||c|c|c|c|c|c|c|c}
  		n & 0 & 1 & 2 & 3 & 4 & 5 & 6 & \cdots \\\hline
  		d_n & \left\lfloor\frac{32}{13}\right\rfloor=2 
  			& \left\lfloor\frac{60}{13}\right\rfloor=4 
  			& \left\lfloor\frac{80}{13}\right\rfloor=6 
  			& \left\lfloor\frac{20}{13}\right\rfloor=1 
  			& \left\lfloor\frac{70}{13}\right\rfloor=5 
  			& \left\lfloor\frac{50}{13}\right\rfloor=3 
  			& \left\lfloor\frac{110}{13}\right\rfloor=8 
  			%& \left\lfloor\frac{60}{13}\right\rfloor=4 
  			& \cdots\\\hline
  		R_n & \frac 6{13} & \frac 8{13} & \frac 2{13} & \frac 7{13} 
  			& \frac 5{13} & \frac{11}{13} & \frac{6}{13} %& \frac 8{13} 
  			& \cdots
  	\end{array}
  \]
  Since $R_6=R_0$, the process repeats and we obtain $D(\frac{32}{13})=2.461538461538\cdots$
  
  \item\begin{enumeratea}
  	\item Formally this requires induction. Informally, it should be clear that $0\le R_0<1$, whence $d_1=\lfloor 10R_0\rfloor$ is an integer $0\le d_0\le 9$.
  	
  	\item $D(x)$ converges trivially by the comparison test: $d_n\, 10^{-n}\le 9\cdot 10^{-n}$. Alternatively, the partial sums form a monotone-up sequence bounded above by $x$.\smallbreak
  	Following the hint, let $E_n=x-\sum_{k=0}^nd_k\,10^{-k}$. We prove by induction that $R_n=10^nE_n$.\par
  	The base case is obvious since $R_0=x-d_0=10^0E_0$.\par
  	For the induction step, fix $n\in\N_0$ and assume $R_n=10^nE_n$. Then
  	\begin{align*}
  		R_{n+1}&=10R_n-d_{n+1} =10^{n+1}E_n-d_{n+1}\\
  		&=10^{n+1}x-\sum\limits_{k=0}^nd_k\,10^{n-k}-d_{n+1}\\
  		&=10^{n+1}\left(x-\sum\limits_{k=0}^{n+1}d_k\,10^{-k}\right) 
  			=10^{n+1}E_{n+1}
  	\end{align*}
		But now $E_n=\frac 1{10^n}R_n\to 0$, since $0\le R_n<1$ for all $n$.
		
  	\item Following the hint, since $\sum\limits_{l=0}^\infty 10^{-rl} =\frac 1{1-10^{-r}} =\frac{10^r}{10^r-1} \in\Q$, we see that
  	\[d_0.d_1\cdots d_md_{m+1}\cdots d_{m+r}d_{m+1}\cdots d_{m+r}\cdots =\sum_{k=0}^md_k\, 10^{-k}+\left(\sum_{j=1}^rd_{m+j}\,10^{-m-j}\right)\sum_{l=0}^\infty 10^{-rl}\]
  	is rational.\par
  	Conversely, suppose $x=\frac pq$ is rational in lowest terms. At each stage of the algorithm, one subtracts an integer after multiplying by 10. The denominator of $R_n$ is therefore always a divisor of $q$, whence
  	\[
  		R_n=\frac aq\ \text{ where }\ a\in\{0,1,\ldots,q-1\}
  	\]
  	For $n\in\{0,1,\ldots,q\}$ there are only $q$ possible remainders $R_n$: at least one of these must appear twice; $R_i=R_j$ for some $0\le i<j\le q$. Writing $r=j-i$, it follows that
  	\[
  		\forall k\ge i,\ d_{k+r}=d_k
  	\]
  	whence the decimal is eventually periodic.
  	
  	\item That the two series are equal is easy to check via the geometric series formula:
  	\[9\sum_{n=m+1}^\infty 10^{-n}=\frac{9\cdot 10^{-m-1}}{1-1/10}=10^{-m}\]
  	Now suppose that two different decimals are equal: that is
  	\[\sum_{n=0}^\infty d_n\,10^{-n}=\sum_{n=0}^\infty c_n\,10^{-n}\]
  	Suppose $m\in\N_0$ is minimal such that $c_m\neq d_m$ and assume WLOG that $c_m<d_m$. Then
  	\begin{align*}
  		&(d_m-c_m)10^{-m}+\sum_{n=m+1}^\infty d_n\,10^{-n}=\sum_{n=m+1}^\infty c_n\,10^{-n}\\
  		\implies&(d_m-c_m)+\sum_{n=1}^\infty d_{n+m}\,10^{-n}=\sum_{n=1}^\infty c_{n+m}\,10^{-n}
  	\end{align*}
  	Consider the left and right sides of this equation:
  	\begin{description}
  		\item[Left Side] Since $d_m>c_m$, this is \emph{greater than or equal to} 1 with equality if and only if  $d_m=c_m+1$ and \emph{all} $d_{n+m}=0$.
  		\item[Right Side] Since $9\sum_{n=1}^\infty 10^{-n}=1$, the right side is \emph{less than or equal to} 1 with equality if and only if \emph{all} $c_{n+m}=9$.
  	\end{description}
		We conclude that $d_m=c_m+1$, and that
		\[
			n>m\implies  d_n=0,\ c_n=9
		\]
	\end{enumeratea}
	
	\item\begin{enumeratea}
		\item $D(x)$ terminates $\Longrightarrow x=\frac p{2^a5^b}$ is rational in lowest terms. For example, $x=\frac{193}{250} =\frac{193}{2\cdot 5^2}$ has a terminating decimal, namely $0.772$. Here is a general proof.\smallbreak
		By the Theorem, we know that all possible candidates for a terminating decimal must be rational. Thus assume $x=\frac pq$ is rational in lowest terms. Observe that
		\[
			R_0=\frac pq-d_0=\frac{p-d_0q}q
		\]
		is a fraction with denominator $q$. Similarly,
		\[
			R_1=10R_0-\left\lfloor 10R_0\right\rfloor
		\]
		is $10R_0$ \emph{minus an integer}; it is therefore a fraction whose denominator is either $q$, $\frac 12q$, $\frac 15q$ or $\frac 1{10}q$. Iterating this process, we see that $R_n$ is a fraction with denominator
		\[
			q_n=\frac 1{2^a5^b}q
			\quad\text{where}\quad 
			a,b\in\N_0
		\]
		$D(x)$ terminates if and only if some $q_n=1$ (then $R_n=\frac 01=0$), which happens if and only if $x$ has the form described above.
		
		\item Think back to the proof. If $x=\frac pq$ is in lowest terms, then in the first $q+1$ remainders, one remainder must appear at least twice. The seemingly largest period is therefore $q$ (if $R_0=R_q$). However, if any remainder were ever zero, then the decimal terminates (with `period' 1). The longest possible period will therefore be $q-1$, which happens if the remainders $R_0,R_1,\ldots,R_{q-2}$ are distinct and non-zero, and $R_{q-1}=R_0$. The example with $\frac 17$ (period 6) shows this. Similarly,
		\[
			\frac 1{23}=0.\textcolor{red}{0434782608695652173913}0434782609\cdots
		\]
		has period 22.
	\end{enumeratea}
	
	\item\begin{enumeratea}
		\item $[0.02020202\cdots]_3=\frac 2{3^2}[1.010101\cdots]_3=\frac 29\sum\limits_{n=0}^\infty 3^{-2n}=\frac 2{9(1-\frac 19)}=\frac 14$.\\
		
		Following the algorithm,
		\[
			\def\arraystretch{1.4}
			\begin{array}{l||c|c|c|c|c|}
  			n & 0 & 1 & 2 & 3 & \cdots\\\hline
  			d_n & \left\lfloor\frac{1}{2}\right\rfloor=0 
  				& \left\lfloor\frac 32\right\rfloor=1 
  				& \left\lfloor\frac 32\right\rfloor=1 
  				& \left\lfloor\frac 32\right\rfloor=1 
  				& \cdots\\\hline
  			R_n & \frac 12 & \frac 12 & \frac 12 & \frac 12 & \cdots
  		\end{array}
  		\implies \frac 12=[0.11111\cdots]_3
  	\]
  	We could also have done this by multiplying the expression for $\frac 14$ by 2.\smallbreak
  	
  	Now for $\frac 15$:
		\[
			\def\arraystretch{1.4}
			\begin{array}{l||c|c|c|c|c|c|}
  			n & 0 & 1 & 2 & 3 & 4 & \cdots\\\hline
  			d_n & \left\lfloor\frac{1}{5}\right\rfloor=0 
  				& \left\lfloor\frac 35\right\rfloor=0 
  				& \left\lfloor\frac 95\right\rfloor=1 
  				& \left\lfloor\frac{12}5\right\rfloor=2 
  				& \left\lfloor\frac 65\right\rfloor=1
  				& \cdots\\\hline
  			R_n & \frac 15 & \frac 35 & \frac 45 & \frac 25 & \frac 15 & \cdots
  		\end{array}
  	\]
		from which the ternary representation repeats:
		\[
			\frac 15=[0.012101210121\cdots]_3
		\]

		\item The theorem goes through almost unchanged: each $t_n\in\{0,1,2\}$ whenever $n\ge 1$, every ternary expansion converges to $T(x)=x$, rational numbers have eventual periodicity, and terminating ternary expansions have another representation: e.g.
		\[
			[1.2012]_3=[1.20112222222\cdots]_3
		\]
		where the final non-zero term is reduced by 1 and an infinite string of 2's added.
		
		\item Suppose $n=p_1^{\mu_1}\cdots p_k^{\mu_k}$ is the unique prime factorization of $n$. The (positive) real numbers $x$ whose $n$-ary expansion terminates are precisely those rational numbers whose (lowest-term) denominators are divisible by no other primes than $p_1,\ldots,p_n$.\par
		For instance, in base-$60=2^2\cdot 3\cdot 5$, the expansion of $\frac{1001}{450}$ will terminate, but that of $\frac 17$ will not (it is 3-periodic though!). In case you are curious:
		\[
			\frac{1001}{450}=[2;13,28]_{60}=2+\frac{13}{60}+\frac{28}{60^2},\qquad 
			\frac 17=[0;8,34,17,8,34,17,\ldots]_{60}
		\]
	\end{enumeratea}
\end{enumerate}

  

